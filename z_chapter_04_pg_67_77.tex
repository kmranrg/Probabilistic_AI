\documentclass[handout,aspectratio=169]{beamer}
\usepackage[english]{babel}
\usepackage{metalogo}
\usepackage{listings}
\usepackage{fontspec}
\usepackage{tikz}
\usepackage{graphicx}
\usepackage{subcaption}
%%%%%%%%%%%%%%%%%%%%%%%%%%%%%%%%%%%%%%%%%%%%%%%%%%%%%%%%%%%%%%%%%%%%%%%%%%%%%%%%%%%%%%%%
%%%%%%%%%%%%%%%%%%%%%%%%%%%%%%%%%%%%%%%%%%%%%%%%%%%%%%%%%%%%%%%%%%%%%%%%%%%%%%%%%%%%%%%%
%%%%%%%%%%%%%%%%%%%%%%%%%%%%%%%%%%%%%%%%%%%%%%%%%%%%%%%%%%%%%%%%%%%%%%%%%%%%%%%%%%%%%%%%
\usepackage{xcolor}
\definecolor{BK}{HTML}{3b3838}
\definecolor{Blue}{RGB}{88, 105, 225}
\newcommand{\BK}[1]{\color{BK} #1}
\newcommand{\URed}[1]{\color{URed} #1}
\newcommand{\white}[1]{\color{white} #1}
\newcommand{\Orange}[1]{\color{QOrange} #1}
\newcommand{\blue}[1]{\color{blue} #1}
\newcommand{\Blue}[1]{\color{Blue} #1}
\newcommand{\red}[1]{\color{red} #1}
\newcommand{\gray}[1]{\color{gray} #1}
\newcommand{\orange}[1]{\color{orange} #1}
\usepackage{accents}
\newcommand\thickbar[1]{\accentset{\rule{.4em}{.8pt}}{#1}}
\newcommand{\ubar}[1]{\underaccent{\bar}{#1}}
\newcommand{\hatbar}[1]{\bar{\hat{#1}}}
\newcommand{\hatubar}[1]{\underaccent{\bar}{\hat{#1}}}
\newcommand{\ul}[1]{\underline{#1}}
\newcommand{\ol}[1]{\overline{#1}}
%%%%%%%%%%%%%%%%%%%%%%%%%%%%%%%%%%%%%%%%%%%%%%%%%%%%%%%%%%%%%%%%%%%%%%%%%%%%%%%%%%%%%%%%
\newcommand{\bO}[1]{\bf \Orange #1}
%%%%%%%%%%%%%%%%%%%%%%%%%%%%%%%%%%%%%%%%%%%%%%%%%%%%%%%%%%%%%%%%%%%%%%%%%%%%%%%%%%%%%%%%
%%%%%%%%%%%%%%%%%%%%%%%%%%%%%%%%%%%%%%%%%%%%%%%%%%%%%%%%%%%%%%%%%%%%%%%%%%%%%%%%%%%%%%%%
\newcommand{\YBox}[3]{
\begin{center}
\begin{minipage}{#1\linewidth}
\begin{exampleblock}{#2}
    #3
\end{exampleblock}
\end{minipage}
\end{center}
}

\newcommand{\OBox}[3]{
\begin{center}
\begin{minipage}{#1\linewidth}
\begin{block}{#2}
    #3
\end{block}
\end{minipage}
\end{center}
}

\newcommand{\Columns}[4]{
\begin{columns}
\begin{column}{#1\textwidth}
#2
\end{column}
\begin{column}{#3\textwidth}  %%<--- here
#4
\end{column}
\end{columns}
}
\input{commands_math}
\usetheme{Nord}
%%%%% \setmainfont{}
\setmainfont{Montserrat}
\setsansfont{Andika New Basic}
\setmonofont{DejaVu Sans Mono}

%%%%%%%%%%%%%%%%%%%%%%%%%%%%%%%%%%%%%%%%%%%%%%%%%%%%%%%%%%%%%%%%%%%%%%%%%%%%%%%%%%%%%%%%
%%%%%%%%%%%%%%%%%%%%%%%%%%%%%%%%%%%%%%%%%%%%%%%%%%%%%%%%%%%%%%%%%%%%%%%%%%%%%%%%%%%%%%%%
\newcommand{\bs}{

\bigskip

}
\newcommand{\ms}{

\medskip

}
%%%%%%%%%%%%%%%%%%%%%%%%%%%%%
\usepackage{tcolorbox}
\newtcolorbox{mybox}{width=9cm, left=0mm,right = 0mm,top=1mm,bottom=1mm,boxsep=0mm}
\newtcolorbox{myboxL}{width=15cm, left=0mm,right = 0mm,top=1mm,bottom=1mm,boxsep=0mm}
\newtcolorbox{mybox1}{width=5.5cm, left=0mm,right = 0mm,top=1mm,bottom=1mm,boxsep=0mm,colback=BK, coltext=QOrange}

\newtcolorbox{mybox2}{width=12cm, left=0mm,right = 0mm,top=1mm,bottom=1mm,boxsep=0mm,colback=BK, coltext=QOrange}

\newtcolorbox{mybox3}{width=3cm, left=0mm,right = 0mm,top=1mm,bottom=1mm,boxsep=0mm,colback=BK, coltext=QOrange}

\newtcolorbox{mybox4}{width=11cm, left=1mm,right = 1mm,top=1mm,bottom=1mm,boxsep=0mm,colback=BK, coltext=QOrange}
%%%%%%%%%%%%%%%%%%%%%%%%%%%%%%%%%%%%%%%%%%%%%%%%%%%%%%%%%%%%%%%%%%%%%%%%%%%%%%%%%%%%%%%%
%%%%%%%%%%%%%%%%%%%%%%%%%%%%%%%%%%%%%%%%%%%%%%%%%%%%%%%%%%%%%%%%%%%%%%%%%%%%%%%%%%%%%%%%

\usepackage{amsmath}
\graphicspath{{figs/fig00/}}
%-----------------
%	TITLE
%-----------------
\title{
Chapter 04: Gaussian Processes \\ Pages: 67-77
}
\author{\bf Notes by Kumar Anurag}
\date{}

%-----------------
%	SUBTITLE
%-----------------
\subtitle{
Probabilistic Artificial Intelligence
}

\begin{document}

%-----------------
%	TITLE PAGE
%-----------------
{\usebackgroundtemplate{\includegraphics[width=\paperwidth]{chapter_figs/01_figs/titlepic.png}}
	\begin{frame}[plain,noframenumbering]
		\maketitle
	\end{frame}}


%-----------------------
%	PRESENTATION SLIDES
%-----------------------

\begin{frame}{4.4 Model Selection}
\begin{itemize}
    \item So far, we assumed hyperparameters $\boldsymbol{\theta}$ (e.g., kernel parameters) are known.
    \item In practice, we must learn $\boldsymbol{\theta}$ from data.
    \item A common supervised learning technique:
    \begin{itemize}
        \item Pick $\boldsymbol{\theta}$ such that the resulting predictor $\hat{f}_\theta$ performs best on a hold-out validation set.
    \end{itemize}
    \item This is known as \textbf{point estimate-based model selection}.
    \item Later, we’ll contrast it with a fully \textbf{Bayesian approach} that uses the posterior over $f$ instead of a point estimate.
\end{itemize}
\pause
\vspace{0.3cm}

\end{frame}

\begin{frame}{4.4.1 Optimizing Validation Performance}
\begin{itemize}
    \item Split data $\mathcal{D}$ into:
    \begin{itemize}
        \item Training set $\mathcal{D}^{\text{train}} = \{(x^{\text{train}}_i, y^{\text{train}}_i)\}_{i=1}^n$
        \item Validation set $\mathcal{D}^{\text{val}} = \{(x^{\text{val}}_j, y^{\text{val}}_j)\}_{j=1}^m$
    \end{itemize}
    \item Step 1: Train a model $\hat{f}_j$ for some candidate $\theta_j$ on the training set:
    \[
    \hat{f}_j \triangleq \arg\max_f p(f \mid x_{1:n}^{\text{train}}, y_{1:n}^{\text{train}}) \tag{4.22}
    \]
    \item Step 2: Select $\hat{\theta}$ based on validation performance:
    \[
    \hat{\theta} \triangleq \arg\max_{\theta_j} p(y_{1:m}^{\text{val}} \mid x_{1:m}^{\text{val}}, \hat{f}_j) \tag{4.23}
    \]
\end{itemize}
\pause

\end{frame}

\begin{frame}{Why Validation Sets Help: Remark 4.8}
\begin{block}{Approximating Population Risk}
Minimizing loss on the same data used for training often leads to overfitting.

Instead, use independent validation data to estimate risk:
\[
\frac{1}{m} \sum_{i=1}^{m} \ell(y_i^{\text{val}} \mid x_i^{\text{val}}, \hat{f}_j) \approx \mathbb{E}_{(x, y) \sim \mathcal{P}} \left[\ell(y \mid x, \hat{f}_j)\right] \tag{4.24}
\]

\begin{itemize}
    \item This is an empirical approximation of the true population risk.
    \item Helps prevent overfitting and ensures generalization.
\end{itemize}
\end{block}

\pause
\vspace{0.3cm}
\textbf{Limitation:} Still relies on a \textit{point estimate} $\hat{f}_j$. Can we do better?
\end{frame}

\begin{frame}{Maximizing the Marginal Likelihood}
\vspace{-0.2cm}
\begin{itemize}
    \item In Bayesian regression, we don’t want to pick a single function $\hat{f}$.
    \item Instead, we want to score how likely the observed data is under all possible functions defined by kernel parameters $\theta$.
\end{itemize}

\vspace{0.3cm}
\textbf{Marginal likelihood:}
\[
p(y_{1:n} \mid x_{1:n}, \theta) = \int p(y_{1:n} \mid x_{1:n}, f, \theta) \cdot p(f \mid \theta) \, df \tag{4.26}
\]

\pause
\vspace{0.2cm}
\textbf{Key idea:} We integrate over all functions $f$ rather than picking just one.

\vspace{0.3cm}
This gives us a more robust measure of how “compatible” $\theta$ is with the observed data.
\end{frame}


\begin{frame}{Why This Works — Intuition Behind the Math}
\textbf{Marginal likelihood prefers models that:}
\begin{itemize}
    \item Fit the data well \rightarrow  \, high likelihood
    \item Are not overly complex \rightarrow \, high prior probability
\end{itemize}

\begin{columns}
    \column{0.6\textwidth}
    \includegraphics[width=\linewidth]{chapter_figs/04_figs/table.png}
    \column{0.4\textwidth}
    \includegraphics[width=0.8\linewidth]{chapter_figs/04_figs/f48.png}
\end{columns}




\pause
\vspace{0.3cm}
\textbf{Conclusion:} Maximizing marginal likelihood automatically balances these two forces. 
\end{frame}

\begin{frame}{Marginal Likelihood for GPs}
\vspace{-0.2cm}
\textbf{Recall from GP prior:}
\[
y_{1:n} \mid x_{1:n}, \theta \sim \mathcal{N}(0, K_{f,\theta} + \sigma_n^2 I) \tag{4.27}
\]

Let $K_{y,\theta} = K_{f,\theta} + \sigma_n^2 I$

\pause
\vspace{0.3cm}
Then the marginal likelihood becomes:
\[
p(y \mid x, \theta) = \mathcal{N}(y \mid 0, K_{y,\theta})
\]

Taking negative log:
\[
\mathcal{L}(\theta) = \frac{1}{2} y^\top K_{y,\theta}^{-1} y + \frac{1}{2} \log \det K_{y,\theta} + \frac{n}{2} \log 2\pi \tag{4.28}
\]

\pause
\textbf{Breaking this down:}
\begin{itemize}
    \item $\boxed{y^\top K^{-1} y}$ — data fit (how well predictions match $y$)
    \item $\boxed{\log \det K}$ — model complexity penalty
    \item Constant term $\frac{n}{2} \log 2\pi$ — does not depend on $\theta$
\end{itemize}
\end{frame}

\begin{frame}{Marginal Likelihood for GPs}

Drop the constant \rightarrow \, final objective:
\[
\hat{\theta}_{\text{MLE}} = \arg\min_\theta \left(
\frac{1}{2} y^\top K_{y,\theta}^{-1} y + \frac{1}{2} \log \det K_{y,\theta}
\right) \tag{4.29}
\]
\end{frame}

\begin{frame}{Gradient of Log Marginal Likelihood}
To optimize $\mathcal{L}(\theta)$, we need the gradient $\frac{\partial}{\partial \theta_j} \log p(y \mid x, \theta)$.

\vspace{0.3cm}
\textbf{GPs allow this in closed form:}
\[
\frac{\partial}{\partial \theta_j} \log p(y \mid x, \theta)
= \frac{1}{2} \text{tr}\left[
\left( \boldsymbol{\alpha} \boldsymbol{\alpha}^\top - K^{-1}_{y,\theta} \right)
\frac{\partial K_{y,\theta}}{\partial \theta_j}
\right] \tag{4.30}
\]

\pause
\textbf{Where:}
\begin{itemize}
    \item $\boldsymbol{\alpha} = K_{y,\theta}^{-1} y$
    \item $\frac{\partial K_{y,\theta}}{\partial \theta_j}$ is the derivative of the kernel matrix w.r.t. hyperparameter $\theta_j$
    \item $\text{tr}(M)$ = trace of matrix $M$ (sum of diagonal elements)
\end{itemize}

\vspace{0.3cm}
This allows gradient-based optimization (e.g. Adam, SGD).
\end{frame}

\begin{frame}{MAP and Full Bayesian Treatment of $\theta$}
\textbf{MLE:} finds the best $\theta$ by maximizing $p(y \mid x, \theta)$, when no prior knowledge about $\theta$

\pause
\textbf{MAP:} adds a prior over $\theta$:
\[
\hat{\theta}_{\text{MAP}} = \arg\max_\theta \, p(\theta) \cdot p(y \mid x, \theta) \tag{4.31}
\]

Taking log:
\[
= \arg\min_\theta \left(
- \log p(\theta) - \log p(y \mid x, \theta)
\right) \tag{4.32}
\]

\pause
\textbf{Fully Bayesian:} integrate out $\theta$
\[
p(y^* \mid x^*, \mathcal{D}) =
\int \int p(y^* \mid x^*, f) \cdot p(f \mid \mathcal{D}, \theta) \cdot p(\theta) \, df \, d\theta \tag{4.33}
\]

\textbf{Challenge:} This integral is intractable in most practical cases.

\pause
So we often settle for MLE or MAP, or use variational approximations.
\end{frame}

\begin{frame}{4.5 Why Do We Need Approximations?}
\textbf{Key issue: Computational Cost}

\begin{itemize}
    \item Gaussian Processes require inversion of an $n \times n$ matrix.
    \item This costs:
    \[
    \mathcal{O}(n^3)
    \]
    \item For large datasets ($n \sim 10^4$ or more), this becomes very slow.
    \item In contrast, Bayesian linear regression has lower cost:
    \[
    \mathcal{O}(nd^2)
    \quad \text{where } d = \text{input feature dimension}
    \]
    \item Therefore, we look for ways to \textbf{approximate} a GP while preserving performance.
\end{itemize}

\vspace{0.3cm}
\textcolor{gray}{Next: What happens when optimization isn't even guaranteed to find a global solution?}
\end{frame}


\begin{frame}{MLL Surface and Two GP Fits}
\begin{columns}
\column{0.55\textwidth}
\textbf{Top plot:} Contour plot of negative log marginal likelihood (MLL)

\begin{itemize}
    \item Axes: Lengthscale $h$ vs noise std $\sigma_n$
    \item Two '+' marks: local optima found during optimization
    \item Shows that MLL can have multiple optima
\end{itemize}

\pause
\vspace{0.3cm}
\textbf{Bottom plots:} GP regression fits corresponding to the two optima

\begin{itemize}
    \item Left: small lengthscale, flexible model \rightarrow \, smooth interpolation
    \item Right: large noise, rigid model \rightarrow \, explains variation as noise
\end{itemize}

\column{0.4\textwidth}
\includegraphics[width=0.8\linewidth]{chapter_figs/04_figs/f410.png}
\end{columns}

\vspace{0.3cm}
\textbf{Takeaway:} Hyperparameter learning is sensitive to initialization and optimization.
\end{frame}

\begin{frame}{4.5.1 Local Methods}
\begin{itemize}
    \item Forward sampling from a GP requires conditioning on all previous observations.
    \item This becomes computationally expensive as $n$ grows.
    \item One idea: Only condition on points “near” the test input $x$.
    \item For example: keep only points $x'$ such that
    \[
    |k(x, x')| \geq \tau \quad \text{for some threshold } \tau > 0
    \]
    \item This “cuts off the tails” of the kernel — treating distant points as independent.
\end{itemize}

\pause
\vspace{0.2cm}
\textbf{Benefit:} Reduces computation.

\textbf{Caution:} If $\tau$ is too large, we may lose important long-range correlations.

\vspace{0.2cm}
This is an example of a \textbf{sparse approximation} to a GP.
\end{frame}

\begin{frame}{4.5.2 Kernel Function Approximation}
\textbf{Goal:} Approximate the kernel $k(x, x')$ directly using a low-dimensional feature map.

\vspace{0.2cm}
\textbf{Idea:}
\[
k(x, x') \approx \phi(x)^\top \phi(x') \tag{4.34}
\]

Then, apply Bayesian linear regression on $\phi(x)$ instead of GPs.

\vspace{0.3cm}
\textbf{Computational benefit:} Time complexity becomes:
\[
\mathcal{O}(nm^2 + m^3)
\]
where $m$ is the number of features.

\vspace{0.3cm}
\pause
This leads us to \textbf{Random Fourier Features (RFF)}, where:
- The feature map $\phi(x)$ is constructed from sine and cosine of random projections
- Based on Fourier transform theory and Bochner's theorem
\end{frame}

\begin{frame}{Fourier View of Kernels}
\textbf{Euler's identity:}
\[
e^{ix} = \cos x + i \sin x \tag{4.35}
\]

\textbf{Fourier transform:}
\[
\hat{f}(\xi) = \int_{\mathbb{R}^d} f(x) e^{-2\pi i \xi^\top x} \, dx \tag{4.36}
\]
\[
f(x) = \int_{\mathbb{R}^d} \hat{f}(\xi) e^{2\pi i \xi^\top x} \, d\xi \tag{4.37}
\]

\pause
\textbf{Stationary kernel as function of } $x - x'$:
\[
k(x - x') = \int_{\mathbb{R}^d} p(\omega) e^{i \omega^\top (x - x')} d\omega \tag{4.38}
\]
Bochner's theorem: $p(\omega)$ is the spectral density of $k$.

\pause
\textbf{Gaussian kernel’s spectral density:}
\[
p(\omega) = (2h^2 \pi)^{d/2} \exp \left( - h^2 \|\omega\|^2 / 2 \right) \tag{4.39}
\]
\end{frame}


\begin{frame}{Random Fourier Features Approximation}
Use Monte Carlo sampling over $\omega \sim p(\omega)$ and $b \sim \text{Unif}[0, 2\pi]$.

\textbf{Define:}
\[
z_{\omega, b}(x) = \sqrt{2} \cos(\omega^\top x + b)
\]

Then:
\[
k(x, x') \approx z(x)^\top z(x') \tag{4.42}
\]
\[
z(x) = \frac{1}{\sqrt{m}} \begin{bmatrix}
z_{\omega^{(1)}, b^{(1)}}(x) \\
\vdots \\
z_{\omega^{(m)}, b^{(m)}}(x)
\end{bmatrix} \tag{4.43}
\]

\vspace{0.3cm}
\textbf{Theorem 4.13:} This approximation converges uniformly to the true kernel as $m \to \infty$

\textbf{Benefit:} Fast kernel approximation, scalable GPs!
\end{frame}

\begin{frame}{4.5.3 Inducing Point Methods: Motivation}
\textbf{Problem:} Using all $n$ training points in a GP is expensive.

\vspace{0.2cm}
\textbf{Idea:} Use a smaller set of \textbf{inducing points} to summarize the full data.

\[
U \triangleq \{\bar{x}_1, \dots, \bar{x}_k\}
\quad \text{where } k \ll n
\]

\vspace{0.3cm}
\textbf{We model:}
\[
u \triangleq \begin{bmatrix} f(\bar{x}_1) & \cdots & f(\bar{x}_k) \end{bmatrix}^\top
\quad \sim \mathcal{N}(0, K_{UU})
\]

\vspace{0.3cm}
\textbf{Recover full GP using marginalization:}
\[
p(f^*, f) = \int p(f^*, f \mid u) \, p(u) \, du \tag{4.45}
\]

\pause
This lets us reduce computations while preserving most information.
\end{frame}

\begin{frame}{Subset of Regressors}
\textbf{Approximation:} Assume $f$ and $f^*$ are conditionally independent given $u$:
\[
p(f^*, f) \approx \int p(f^* \mid u) \, p(f \mid u) \, p(u) \, du \tag{4.46}
\]

\pause
\textbf{From Gaussian conditionals:}
\begin{align*}
p(f \mid u) &\sim \mathcal{N}(K_{AU} K_{UU}^{-1} u, \, K_{AA} - Q_{AA}) \tag{4.47a} \\
p(f^* \mid u) &\sim \mathcal{N}(K_{\ast U} K_{UU}^{-1} u, \, K_{\ast\ast} - Q_{\ast\ast}) \tag{4.47b}
\end{align*}

Where $Q_{ab} = K_{aU} K_{UU}^{-1} K_{Ub}$

\pause
\textbf{Subset of Regressors (SoR):} keep the mean and drop the variances/covariances
\begin{align*}
q_{\text{SoR}}(f \mid u) &= \mathcal{N}(K_{AU}K_{UU}^{-1}u, 0) \tag{4.48a} \\
q_{\text{SoR}}(f^* \mid u) &= \mathcal{N}(K_{\ast U}K_{UU}^{-1}u, 0) \tag{4.48b}
\end{align*}
\textbf{Drawback}: Since we drop the uncertainties in this approach, therefore, we may unrealistic predictions.
\end{frame}

\begin{frame}{Fully Independent Training Conditional}
\pause
\textbf{FITC (Fully Independent Training Conditional):}
\begin{align*}
q_{\text{FITC}}(f \mid u) &= \mathcal{N}(K_{AU}K_{UU}^{-1}u, \, \text{diag}(K_{AA} - Q_{AA})) \tag{4.49a} \\
q_{\text{FITC}}(f^* \mid u) &= \mathcal{N}(K_{\ast U}K_{UU}^{-1}u, \, \text{diag}(K_{\ast\ast} - Q_{\ast\ast})) \tag{4.49b}
\end{align*}

\pause
\textbf{Summary:}
\begin{itemize}
    \item SoR: keeps mean, discards all variance
    \item FITC: keeps mean + variances, ignores off-diagonal covariances
\end{itemize}
\end{frame}

\begin{frame}{Graph: SoR vs FITC}
\begin{figure}
    \centering
    \includegraphics[width=0.3\linewidth]{chapter_figs/04_figs/sor_fitc.png}
\end{figure}

    
\end{frame}

\begin{frame}{Discussion: GP Approximations in Practice}
\begin{itemize}
    \item GPs are flexible non-parametric models that reason over functions.
    \item But exact inference becomes costly with large datasets.
\end{itemize}

\vspace{0.3cm}
\textbf{This chapter introduced several ways to approximate GPs:}
\begin{itemize}
    \item \textbf{Local methods:} Use only nearby data points
    \item \textbf{Kernel approximations:} RFF via Bochner’s theorem
    \item \textbf{Inducing points:} Summarize data through select locations
\end{itemize}

\vspace{0.2cm}
\textbf{Next:} We explore how to combine these probabilistic ideas with scalable models.
\end{frame}














{\usebackgroundtemplate{\includegraphics[width=\paperwidth]{chapter_figs/03_figs/thankyou.png}}
	\begin{frame}[plain,noframenumbering]
	\end{frame}}



\end{document}